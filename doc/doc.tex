\documentclass[a4paper,12pt]{article}

% PACKAGES
\usepackage[utf8]{inputenc}
\usepackage[brazil]{babel}
% math packages
\usepackage{amsmath}
\usepackage{amstext}

\usepackage{url}

\usepackage{multicol}
\usepackage{graphicx}

%
\graphicspath{ {./img/} }
\setlength{\parskip}{1.3ex plus 0.5ex minus 0.3ex}

% HEADER
\title{Trabalho Prático 1 - Truco}
\author{
    Arthur Xavier\\
    \texttt{xavier@dcc.ufmg.br}
    \and
    Jota Júnior\\
    \texttt{}
}
\date{11 de Maio de 2016}

% DOCUMENT
\begin{document}

% TITLE
\maketitle

%%%%%%%%%%%%%%%%%%%%%%%%%%%%%%%%%%%%%%%%%%%%%%%%%%%%%%%%%%%%%%%%%%%%%%%%%%%%%%%
% INTRODUÇÃO
%%%%%%%%%%%%%%%%%%%%%%%%%%%%%%%%%%%%%%%%%%%%%%%%%%%%%%%%%%%%%%%%%%%%%%%%%%%%%%%
\section{Introdução}

O jogo de Truco é um dos jogos de cartas mais jogados no Brasil na atualidade, especialmente entre os mais jovens e universitários. Este trabalho busca, através dos conceitos aprendidos em sala de aula, implementar um jogo de truco funcional para a fixação e práticas destes conceitos.

\subsection{Regras do jogo}
O jogo de Truco é bastante simples, apesar de possuir muitas regras em comparação com outros jogos também populares. Uma partida é jogada por duas equipes, formadas cada uma por um ou dois jogadores. No caso deste trabalho optamos por simplificar a implementação, utilizando apenas equipes de um jogador. Desta forma a \textbf{partida} é jogada por dois jogadores que buscam atingir \textbf{12 pontos} (chamados também \emph{tentos}) para vencer.

\subsubsection{A partida}
Uma partida é jogada em várias \textbf{mãos}. Em uma mão, cada jogador recebe 3 cartas de um baralho constituído pelos cartas de 1 a 7, a dama (no valor de 8), o valete (no valor de 9) e o rei (no valor de 10). Uma mão é jogada em 3 \textbf{rodadas}. O jogador que jogar a carta de valor mais alto vence a rodada. E a equipe que vencer duas rodadas vence a mão. Se todas as rodadas empatarem, nenhuma das equipes ganha ponto e uma nova mão se inicia. Uma partida é jogada em quantas mãos forem necessárias para que um jogador vença.

\subsubsection{Pontuação}
Inicialmente, o vencedor de uma mão ganha 2 tentos. O jogo conta, porém, com um sistema de apostas por mão. Estas apostas são chamadas \textbf{trucos}. Na sua vez de jogar durante uma rodada, um jogador pode fazer uma aposta, isto é, \emph{pedir truco}. Caso isto aconteça, a equipe adversária deve decidir se vai aceitar, fugir ou \emph{pedir 6}. Se eles aceitarem, a mão passa a valer 4 e continua normalmente. Se não aceitarem, a equipe que pediu o truco vence a mão e ganha seus pontos (dois). Se a equipe adversária \emph{pedir 6}, a equipe que pediu o truco deve decidir se vai aceitar, fugir ou \emph{pedir 10}, e assim por diante. Um jogador só pode aumentar o valor da aposta na seguinte ordem: 4, 6, 10 e 12. Por exemplo, ele não pode pedir 10 se o jogo está valendo 4 nem pedir 6 se o jogo vale 2. Uma equipe não pode pedir truco duas vezes seguidas. Quando uma equipe fugir, a outra equipe leva a mão e o valor atual dela em pontos.


%%%%%%%%%%%%%%%%%%%%%%%%%%%%%%%%%%%%%%%%%%%%%%%%%%%%%%%%%%%%%%%%%%%%%%%%%%%%%%%
% Implementação
%%%%%%%%%%%%%%%%%%%%%%%%%%%%%%%%%%%%%%%%%%%%%%%%%%%%%%%%%%%%%%%%%%%%%%%%%%%%%%%
\section{Implementação}


%%%%%%%%%%%%%%%%%%%%%%%%%%%%%%%%%%%%%%%%%%%%%%%%%%%%%%%%%%%%%%%%%%%%%%%%%%%%%%%
% TESTES
%%%%%%%%%%%%%%%%%%%%%%%%%%%%%%%%%%%%%%%%%%%%%%%%%%%%%%%%%%%%%%%%%%%%%%%%%%%%%%%
\section{Testes}


%%%%%%%%%%%%%%%%%%%%%%%%%%%%%%%%%%%%%%%%%%%%%%%%%%%%%%%%%%%%%%%%%%%%%%%%%%%%%%%
% CONCLUSÃO
%%%%%%%%%%%%%%%%%%%%%%%%%%%%%%%%%%%%%%%%%%%%%%%%%%%%%%%%%%%%%%%%%%%%%%%%%%%%%%%
\section{Conclusão}


%%%%%%%%%%%%%%%%%%%%%%%%%%%%%%%%%%%%%%%%%%%%%%%%%%%%%%%%%%%%%%%%%%%%%%%%%%%%%%%
% BIBLIOGRAFIA
%%%%%%%%%%%%%%%%%%%%%%%%%%%%%%%%%%%%%%%%%%%%%%%%%%%%%%%%%%%%%%%%%%%%%%%%%%%%%%%
\section{Bibliografia}
http://www.megajogos.com.br/jogosonline/truco-mineiro/regras

\end{document}
